\section{FAQ and Common Installation Problems}
\label{sec:faq}

\begin{comment} 

Erro de quando se abre um modelo sem ter executado a transformacao antes. Os
 ficheiros ainda n tao carregados na plataforma\ldots
 
 Solucao: adicionar atributo no modelo (nao viavel). Outra: Correr simplesmente
 a transformacao

Erro de inicializacao de prolog e isso
	Solucao: rever as path variables e copiar o jpl.jar pra lib do java. Verificar
	tbm se a versao usada do java eh a correcta.

% Erro de nome de package mal escrito.

% Erro de file not found 

% Erro de association nao declarada no metamodelo

% Erro de classe nao declarada no metamodel

\end{comment}

\subsection{SWI-Prolog: [FATAL Error: Could not find system resources]}

This error occurs when you one (or more) of the following steps in the section \ref{sec:installation}:
\begin{itemize}
\item Install swi prolog.
\item Add the bin directory of prolog to the \emph{Path} variable.
\item Set the \emph{SWI\_HOME\_DIR} system variable.
\item Copy the jpl.jar file to the appropriate destination.
\end{itemize}

\subsection{UnsatisfiedLinkError: no jpl in java.library.path}

This error can be caused by many things. One of then is an incorrect \emph{PATH} \textbf{system} environment variable.
Make sure you have the following two items in your \textbf{system} \emph{PATH}:
\begin{itemize}
	\item \verb=C:\Program Files (x86)\pl\bin= (or your own path to prolog's bin
	directory);
	\item \verb=C:\Program Files (x86)\Java\jre1.8.0_25\bin= (or your own path
	to java's bin directory);
\end{itemize}

You need to restart your computer after updating the environment variable.

If this does not solve your problem, review the installation steps in the section \ref{sec:installation}.

\subsection{Prolog fatal error: EXCEPTION\_ACCESS\_VIOLATION}

We had this problem with the 64 bits version of Prolog.
Uninstall your current prolog and install a newer version.















